\documentclass[]{elsarticle} %review=doublespace preprint=single 5p=2 column
%%% Begin My package additions %%%%%%%%%%%%%%%%%%%
\usepackage[hyphens]{url}

  \journal{NeuroImage} % Sets Journal name


\usepackage{lineno} % add
\providecommand{\tightlist}{%
  \setlength{\itemsep}{0pt}\setlength{\parskip}{0pt}}

\bibliographystyle{elsarticle-harv}
\biboptions{sort&compress} % For natbib
\usepackage{graphicx}
\usepackage{booktabs} % book-quality tables
%%%%%%%%%%%%%%%% end my additions to header

\usepackage[T1]{fontenc}
\usepackage{lmodern}
\usepackage{amssymb,amsmath}
\usepackage{ifxetex,ifluatex}
\usepackage{fixltx2e} % provides \textsubscript
% use upquote if available, for straight quotes in verbatim environments
\IfFileExists{upquote.sty}{\usepackage{upquote}}{}
\ifnum 0\ifxetex 1\fi\ifluatex 1\fi=0 % if pdftex
  \usepackage[utf8]{inputenc}
\else % if luatex or xelatex
  \usepackage{fontspec}
  \ifxetex
    \usepackage{xltxtra,xunicode}
  \fi
  \defaultfontfeatures{Mapping=tex-text,Scale=MatchLowercase}
  \newcommand{\euro}{€}
\fi
% use microtype if available
\IfFileExists{microtype.sty}{\usepackage{microtype}}{}
\ifxetex
  \usepackage[setpagesize=false, % page size defined by xetex
              unicode=false, % unicode breaks when used with xetex
              xetex]{hyperref}
\else
  \usepackage[unicode=true]{hyperref}
\fi
\hypersetup{breaklinks=true,
            bookmarks=true,
            pdfauthor={},
            pdftitle={Recommendations for Processing Head CT Data},
            colorlinks=false,
            urlcolor=blue,
            linkcolor=magenta,
            pdfborder={0 0 0}}
\urlstyle{same}  % don't use monospace font for urls

\setcounter{secnumdepth}{0}
% Pandoc toggle for numbering sections (defaults to be off)
\setcounter{secnumdepth}{0}
% Pandoc header



\begin{document}
\begin{frontmatter}

  \title{Recommendations for Processing Head CT Data}
    \author[JHSPH]{John Muschelli}
   \ead{jmusche1@jhu.edu} 
  
      \address[JHSPH]{Johns Hopkins Bloomberg School of Public Health, Department of
Biostatistics, 615 N Wolfe St, Baltimore, MD, 21205}
    \address[JHMI]{Johns Hopkins Hospital, Department of Neurology, 601 N Caroline St,
Baltimore, MD 21205}
    \address[BIOS]{Brain Injury Outcomes, Johns Hopkins University, 750 East Pratt Street,
Baltimore, MD 21202}
  
  \begin{abstract}
  This is the abstract.
  
  It consists of two paragraphs.
  \end{abstract}
  
 \end{frontmatter}

\hypertarget{introduction}{%
\section{Introduction}\label{introduction}}

Many research applications of neuroimaging use magnetic resonance
imaging (MRI). MRI allows researchers to study a multitude of
applications and diseases, including studying healthy controls. Clinical
imaging, however, relies heavily on X-ray computed tomography (CT) scans
for diagnosis and prognosis. Studies using CT scans cannot generally
recruit healthy controls or large non-clinical populations due to the
radiation exposure and lack of substantial benefit. As such, much of
head CT data is gathered from prospective clinical trials or
retrsopective studies based on health medical record data and hospital
picture archiving and communication system (PACS). We wish to provide
some recommendations and guidelines from our experience with CT data, as
well as insights from working with MRI studies. We will discuss existing
software options for neuroimaging in general and those that are specific
to CT throughout the paper.

We will focus on aspects of quantitatively analyzing the CT data and
getting the data into a format familiar to most MRI neuroimaging
researchers. Therefore, we will not go into detail of DICOM reading
tools or imaging suites for radiologists, which are generally
proprietary and quite costly. Moreover, we will be focussing
specifically on non-contrast head CT data, though many of the
recommendations and software is applicable to images of other areas of
the body.

\hypertarget{data-organization}{%
\section{Data Organization}\label{data-organization}}

Most of the data coming from a PACS is in DICOM (Digital Imaging and
Communications in Medicine) format. Generally, DICOM files are a
combination of metadata about the image (also called a header) and the
individual pixel data, many times embedded in a JPEG format. The header
has a collection of information, usually referred to as fields or tags.
Tags are usually defined by a set of 2 hexadecimal numbers, which are
embedded as 4 alphanumeric characters. For example, \texttt{(0008,103E)}
denotes the \texttt{SeriesDescription} tag for a DICOM file. Most DICOM
readers extract and use these tags for filtering and organizing the
files.

We will use the phrase scanning session (as opposed to ``study'' and
reserve study to denote a trial or analysis), a series for an individual
scan, and a slice for an individual picture of the brain. Each series
(\texttt{Series\ Instance\ UID} tag) and scanning session
(\texttt{Study\ Instance\ UID} tag) should have a unique value in the
DICOM header that allows DICOM readers to organize the data by scanning
session and series.

\hypertarget{dicom-anonymization}{%
\subsection{DICOM Anonymization}\label{dicom-anonymization}}

One of the common issues with DICOM data is that a large amount of
protected health information (PHI) can be contained in the header. DICOM
is a standard where individual fields in the header are to contain the
same values across different scanners and sites, but only if that
manufacturer and site are diligent to ascribing to the DICOM standard.
Though many DICOM header fields are consistent across neuroimaging
studies, duplicate data and additional fields may be required to obtain
the full amount of data required for analysis. Moreover, different
scanning manufacturers can embed information in non-standard fields. The
goal is to remove these fields if they contain PHI but retain these
fields if they embed relevant information of the scan for analysis.
These fields then represent a challenge to a ``lossless'' anonymization
if the data do not conform to a standard across scanning sites,
manufacturers, or protocols.

We will discuss reading in DICOM data and DICOM header fields in the
next section. Though thse steps can be crucial for extracting
information from the data, many times the data must be shared or
transferred before analysis. Depending on the parties receiving the
data, anonymization of the data must be done first. Aryanto, Oudkerk,
and Ooijen (2015) provides a look at a multitude of options for DICOM
anonymization and recommends the RSNA MIRC Clinical Trials Processor
(CTP, https://www.rsna.org/research/imaging-research-tools)
cross-platform Java software as well as the DICOM library
(https://www.dicomlibrary.com/) upload service. We also recommend the
\href{https://www.dclunie.com/pixelmed/software/webstart/DicomCleanerUsage.html}{DicomCleaner}
cross-platform Java program as it has fit many of our needs. Bespoke
solutions can be generated using \texttt{dcm4che} (such as
\texttt{dcm4che-deident}) and other DICOM reading tools (discussed
below), but many of these tools have built-in capabilities that are
difficult to add (such as removing PHI embedded in the pixel data).

\hypertarget{a-note-on-de-identification-time-between-scans}{%
\subsubsection{A note on de-identification: time between
scans}\label{a-note-on-de-identification-time-between-scans}}

Although most of the presented solutions are good at anonymization and
de-identification of the header information, only a few such as CTP,
have the utilities required for longitudinal preservation of date
differences. Though it can be debated whether dates are identifiable
information, some clinical trials and other studies rely on serial CT
imaging data, and the differences between times are crucial to determine
when events occur.

\hypertarget{reading-dicom-data}{%
\subsubsection{Reading DICOM data}\label{reading-dicom-data}}

We will focus on 2 analysis platforms for statistical analysis,
including \texttt{R} (CITE) and \texttt{Python} as well as standalone
software. The main reasons are that \texttt{R} and \texttt{Python} are
free, open source, and we have a working knowledge of the utilities
these have. We are also involved in the Neuroconductor project
(https://neuroconductor.org/) (Muschelli et al. 2018), which is a
repository of \texttt{R} packages for medical image analysis. Other
imaging platforms such as the Insight Segmentation and Registration
Toolkit (ITK) are great pieces of software that can perform many of the
operations that we will be discussing. Moreover, \texttt{MATLAB} has an
extensive general imaging suite, as well as large neuroimaging platforms
such as SPM (CITE). We will touch on some of this software with varying
levels. We aim to present software that we have had used directly for
analysis or preprocessing. Also, other papers and tutorials discuss
their use (CITE).

For reading DICOM data, there are multiple options. The MATLAB imaging
toolbox, \texttt{oro.dicom} \texttt{R} package, \texttt{pydicom}, and
\texttt{ITK} cinterfaces can read DICOM data amongst others. The DICOM
toolkit \texttt{dcmtk} has multiple DICOM manipulation tools, including
\texttt{dcmconv} to convert DICOM files to other imaging formats.

Though most imaging analysis tools can read in DICOM data, there are
downsides to using the DICOM format. In most cases, a DICOM image is
split into a series of slices, where each slice is a different file.
This separation can be cumbersome on data organization if using folder
structures. As noted above, these files also can contain a large amount
of PHI. Some formats may be compressed using proprietary compression
such as JPEG2000; alternatively, if data are not compressed file storage
is inefficient. Most importantly though, many imaging analyses perform
3-dimensional operations, such as smoothing. Thus, putting the data into
a different format may be helpful.

\hypertarget{converting-dicom-to-nifti}{%
\subsection{Converting DICOM to NIfTI}\label{converting-dicom-to-nifti}}

Many different general 3D medical imaging formats exist, such as
ANALYZE, NIfTI, NRRD, and MNC. We recommend the NIfTI format as it can
be read by nearly all medical imaging platforms, has been widely used,
has a format standard, can be stored in a compressed format, and is how
much of the data is released online. Moreover, we will present specific
software to convert DICOM data and the recommended software
(\texttt{dcm2niix}) outputs data in a NIfTI file.

Although we recommend this software, many good and complete solutions
exist. Examples include \texttt{dicom2nifti} in the \texttt{oro.dicom}
\texttt{R} package, \texttt{pydicom}, \texttt{dicom2nifti} in
\texttt{MATLAB}, and using large imaging suites such as using
\texttt{ITK} image reading functions for DICOM files and can write NIfTI
outputs. We recommend the \texttt{dcm2niix} (LINK) function from Chris
Rorden for CT data. The reasons are 1) it works with all major scanners,
2) incorporates gantry-tilt correction for CT data, 3) can handle
variable slice thickness, 4) is open-source, 5) is fast, 6) has
responsive developers, and 7) works on all 3 major operating systems
(Linux/OSX/Windows) (TRUE??). Moreover, the popular AFNI neuroimaging
suite includes a \texttt{dcm2niix} program with its distribution. In
\texttt{R}, the \texttt{divest} package (LINK) and the \texttt{XXXXXXXX}
Python module wraps the underlying code for \texttt{dcm2niix} to provide
the same functionality of \texttt{dcm2niix}, along with the ability to
manipulate and subset the header data as necessary.

We will describe a few of the features above. In some head CT scans, the
gantry is tilted to reduce radiation exposure to non-brain areas, such
as the eyes. Thus, the slices of the image are at an oblique angle. If
slice-by-slice analyses an affine registration (as this tilting is a
shearing) are done, this tilting is not an issue. This tilting does
cause issues for 3D operations as the distance of the voxels between
slices is not correct and especially can show odd visualizations
(FIGURE). The \texttt{dcm2niix} output returns both the corrected and
non-corrected image. As the correction moves the slices to a different
area, \texttt{dcm2niix} may pad the image so that the entire head is
still inside the field of view. As such, this may cause issues with
algorithms that require the 512x512 axial slice dimensions. Though less
common, variable slice thickness can occur in reconstructions where only
a specific area of the head is of interest. For example, an image may
have 5mm slice thicknesses throughout the image, except for areas near
the third ventricle, which has a 2.5mm slice thickness. To correct for
this, \texttt{dcm2niix} interpolates between slices to ensure each image
has a consistent voxel size. Again, \texttt{dcm2niix} returns both the
corrected and non-corrected image.

Once converted to NIfTI format, one should ensure the scale of the data.
Most CT data is betweeen \(-1024\) and \(3071\) HU. Values less than
\(-1024\) are commonly found due to areas of the image outside the field
of view that were not actually imaged. One first processing step would
be to Winsorize the data to the {[}\(-1024\), \(3071\){]} range. After
this step, the header elements \texttt{scl\_slope} and
\texttt{scl\_inter} elements of the NIfTI image should be set to \(1\)
and \(0\), respectively, to ensure no data rescaling is done in other
software. Though HU is the standard format used in CT analysis, negative
HU values may causes issues with standard imaging pipelines built for
MRI, which typically have positive values. Rorden (CITE) proposed a
lossless transformation, called Cormack units, which have a minimum
value of \(0\).

\hypertarget{brain-extraction-in-ct}{%
\subsection{Brain Extraction in CT}\label{brain-extraction-in-ct}}

Head CT data typically contains the subject's head, face, and maybe neck
and other lower structures, depending on the field of view.
Additionally, other artifacts are typically present, such as the pillow
the subject's head was on, the bed/girney, and any instruments in the
field of view. We do not provide a general frameowrk to extract the
subject from the artifact data, but provide some recommendations for
working heuristics. Typically the range of data for a subject is within
\(-100\) to \(300\), excluding the skull, other bones, and
calcificiations. Creating a mask from this data range tends to remove
the bed/girney, most intstruments, the pillow, and the background.
Retaining the largest connected component, filling holes (to include the
skull), and masking the original data with this resulting mask will
return the subject. Note, care must be taken whenever a masking
procedure is used with HU values as \(0\) is a real value: if all values
are set to \(0\) outside the mask in an image with HU values, the value
of \(0\) corresponds to both \(0\) HU and outside of mask. Either
transforming the data into Cormack units, adding a value to the data
(such as \(1025\)) then setting values to \(0\), or using \texttt{NaN}
are recommended in negative values are of interest.

One of the most common steps in processing imaging of the brain is to
remove non-brain structures from the image. We have published a method
that uses the brain extraction tool (BET) from FSL, originally built for
MRI, to perform brain extraction (CITE) with code provided
(http://bit.ly/CTBET\_BASH). Many papers present brain extracted CT
images, but do not always disclose the method of extraction. Recently,
convolutional neural networks and shape propagation techniques have been
quite successful in this task (Akkus et al. 2018) and models have been
released (https://github.com/aqqush/CT\_BET). Overall, much research can
still be done in this area as conditions such as traumatic brain injury
(TBI) and surgery, such as craniotomies or craniectomies can cause these
methods to potentially fail.

\hypertarget{registration-to-a-ct-template}{%
\subsection{Registration to a CT
template}\label{registration-to-a-ct-template}}

Though many analyses in clinical data may be subject-specific,
population-level analyses are still of interest. In some cases,
registration from a template space to a subject space can provide
information that can be aggregated across people for analysis. For
example, one can perform a label fusion approach to CT data to infer the
size of the hippocampus and then analyze hippocampi sizes across the
population. One issue with these approaches is that most templates and
approaches rely on an MRI template. These templates were developed by
taking MRIs of healthy volunteers, which is unethical with CT data due
to the radiation exposure. To create templates, retrospective searches
through medical records can provide patients who came in with symptoms
warranting a CT scan, such as a headache, but had a diagnosis of no
pathology or damage. Thus, these neuro-normal scans are similar to that
of those collected those in MRI research studies but with some important
differences. As these are retrospective, inclusion criteria information
may not be easily obtainable if not clinically collected, scanning
protocols and parameters may vary, even within hospital and especially
over time, and these patients still have neurological symptoms. Though
these challenges exist, with a large enough patient population and a
research consent at an institution, these scans can be used to create
templates and atlases based on CT. To our knowledge, the first publicly
available head CT template exists was released in 2012 by Rorden et al.
(2012), for the purpose of spatial normalization.

One interesting aspect of CT image registration is again that CT data
has units within the same range. To say they are uniformly standardized
is a bit too strong in our opinion, but you can think of them as much
more standardized than MRI due to the nature of the data. This
standardization may warrant or allow the user different search and
evaluation cost functions for registration, such as least squares. We
have found though that normalized mutual information still performs well
in CT-to-CT registration and should be at least considered when using
CT-to-MRI or CT-to-PET registration. Along with the template above,
Rorden et al. (2012) released the clinical toolbox
(https://github.com/neurolabusc/Clinical) for SPM {[}CITE{]} to allow
researchers to register head CT data to a standard space. However, as
the data are in NIfTI format, almost all image registration software
should work, though one should consider transforming the units using
Cormack units or other transformations as negative values may implicitly
be excluded in some software built for MRI registration.

\hypertarget{publicly-available-data}{%
\section{Publicly Available Data}\label{publicly-available-data}}

With the issues of PHI above coupled with the fact that most CT data is
acquired clinically and not in a research setting, there is a dearth of
publicly available data for head CT compared to head MRI. Sites for
radiological training such as Radiopedia (https://radiopaedia.org/) have
many cases of head CT data, but these are converted from DICOM to
standard image formats (e.g.~JPEG) so crucial information such as
Hounsfield Units and pixel dimensions are lost.

Large repositories of head CT data do exist, though, and many in DICOM
format, with varying licenses and uses. The CQ500 (Chilamkurthy et al.
2018) dataset provides approximately 500 head CT scans with different
clinical pathologies and diagnoses, with a non-commercial license. The
Cancer Imaging Archive (TCIA) has hundreds of CT scans, many cases with
brain cancer. TCIA also has a RESTful API, which allows cases to be
downloaded in a scripted way; for example, the \texttt{TCIApathfinder} R
package (Russell 2018) and Python \texttt{tciaclient} module provide an
interface. The Stroke Imaging Repository Consortium
(http://stir.dellmed.utexas.edu/) also has head CT data available for
stroke. The National Biomedical Imaging Archive (NBIA,
https://imaging.nci.nih.gov) demo provides some head CT data, but are
duplicated from TCIA. The NeuroImaging Tools \& Resources Collaboratory
(NITRC, https://www.nitrc.org/) provides links to many data sets and
tools, but no head CT data at this time. The RIRE (Retrospective Image
Registration Evaluation, http://www.insight-journal.org/rire/) and MIDAS
(http://www.insight-journal.org/midas) projects have a small set of
publicly available head CT.

\hypertarget{pipeline}{%
\subsection{Pipeline}\label{pipeline}}

Overall, our recommended pipeline is as follows: 1. Use CTP to organize
and anonymize the DICOM data from a PACS. 2. Extract relevant header
information for each DICOM, using software such as \texttt{dcmdump} from
\texttt{dcmtk} and store, excluding PHI. 3. Convert DICOM to NIfTI using
\texttt{dcm2niix}, which can create brain imaging data structure (BIDS)
formatted data. Use the tilt-corrected and data with uniform voxel size.

After, depending on the purpose of the analysis, you may do registration
then brain extraction, brain extraction then registration, or not do
registration at all. If you are doing analysis of the skull, you can
also use brain extraction as a first step to identify areas to be
removed.

For brain extraction, run \texttt{BET} for CT or \texttt{CT\_BET}
(especially if you have GPUs for the neural network).

\hypertarget{concurrent-mri}{%
\subsection{Concurrent MRI}\label{concurrent-mri}}

Additionally, the spatial constrast is much lower than T1-weighted MRI
for image segmentation. Therefore, concurrent MRI may be useful. One
large issue

\hypertarget{conclusions}{%
\section{Conclusions}\label{conclusions}}

We present a simple pipeline for preprocessing of head CT data, along
with software options of reading and transforming the data. We have
found that though many tools exist for MRI and are applicable to CT
data. Noticeable differences exist between the data in large part due to
the collection setting (research vs.~clinical), data access, data
organization, and population-level data. As CT scans provide fast and
clinically relevant information and with the increased interest in
machine learning in medical imaging data, particularly deep learning
using convolutional neural networks, research and quantitative analysis
of head CT data is bound to increase. We believe this presents an
overview of a useful set of tools and data for research in head CT.

For research using head CT scans to have the level of interest and
success as MRI, additional publicly available data needs to be released.
We saw the explosion of research in MRI, particularly functional MRI, as
additional data were released and and consortia created truly
large-scale studies. This collaboration is possible at an individual
institution, but requires scans to be released from a clinical
population, where consent must be first obtained, and upholding patient
privacy must be a top priority. Large internal data sets likely exist,
but institutions need incentives to release these data sets. Also,
though institutions have large amounts of rich data, general methods and
applications require data from multiple institutions as parameters,
protocols, and population characteristics can vary widely across
institutions.

One of the large hurdles after creating automated analysis tools or
supportive tools to help radiologists and clinicians is the
reintegration of this information into the healthcare system. We do not
present answers to this difficult issue, but note that these tools need
to be created to show cases where this reintegration can improve patient
care, outcomes, and other performance metrics. We hope the tools and
discussion we have provided advances those efforts for researchers
starting in this area.

\hypertarget{references}{%
\section*{References}\label{references}}
\addcontentsline{toc}{section}{References}

\hypertarget{refs}{}
\leavevmode\hypertarget{ref-ct_bet}{}%
Akkus, Zeynettin, Petro M Kostandy, Kenneth A Philbrick, and Bradley J
Erickson. 2018. ``Extraction of Brain Tissue from CT Head Images Using
Fully Convolutional Neural Networks.'' In \emph{Medical Imaging 2018:
Image Processing}, 10574:1057420. International Society for Optics;
Photonics.

\leavevmode\hypertarget{ref-aryanto2015free}{}%
Aryanto, KYE, M Oudkerk, and PMA van Ooijen. 2015. ``Free DICOM
de-Identification Tools in Clinical Research: Functioning and Safety of
Patient Privacy.'' \emph{European Radiology} 25 (12): 3685--95.

\leavevmode\hypertarget{ref-cq500}{}%
Chilamkurthy, Sasank, Rohit Ghosh, Swetha Tanamala, Mustafa Biviji,
Norbert G Campeau, Vasantha Kumar Venugopal, Vidur Mahajan, Pooja Rao,
and Prashant Warier. 2018. ``Deep Learning Algorithms for Detection of
Critical Findings in Head Ct Scans: A Retrospective Study.'' \emph{The
Lancet} 392 (10162): 2388--96.

\leavevmode\hypertarget{ref-neuroconductor}{}%
Muschelli, J., A. Gherman, J. P. Fortin, B. Avants, B. Whitcher, J. D.
Clayden, B. S. Caffo, and C. M. Crainiceanu. 2018. ``Neuroconductor: An
R Platform for Medical Imaging Analysis.'' \emph{Biostatistics},
January.

\leavevmode\hypertarget{ref-rorden2012age}{}%
Rorden, Christopher, Leonardo Bonilha, Julius Fridriksson, Benjamin
Bender, and Hans-Otto Karnath. 2012. ``Age-Specific Ct and Mri Templates
for Spatial Normalization.'' \emph{Neuroimage} 61 (4): 957--65.

\leavevmode\hypertarget{ref-TCIApathfinder}{}%
Russell, Pamela. 2018. \emph{TCIApathfinder: Client for the Cancer
Imaging Archive Rest Api}.
\url{https://CRAN.R-project.org/package=TCIApathfinder}.

\end{document}


